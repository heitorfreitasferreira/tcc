% main_ppgco_ufu.tex v1.0, Lásaro Camargos e Denise Guliato
% adaptado de modeloABNT2.tex, v1.0 athila 
% ------------------------------------------------------------------------
% ------------------------------------------------------------------------
% eesc: Modelo de Trabalho Acadêmico (tese de doutorado, dissertação de
% mestrado e trabalhos monográficos em geral) em conformidade com 
% ABNT NBR 14724:2011. Esta classe estende as funcionalidades da classe
% abnTeX2 elaborada de forma a adequar os parâmetros exigidos pelas 
% normas USP e do departamento de elétrica da Escola de Engenharia 
% de São Carlos - USP.
% ------------------------------------------------------------------------
% ------------------------------------------------------------------------

% ------------------------------------------------------------------------
% Opções:
% 	tesedr:     Formata documento para tese de doutorado
%	qualidr:    Formata documento para qualificação de doutorado
% 	dissertmst: Formata documento para dissertação de mestrado
% 	qualimst:   Formata documento para qualificação de mestrado
% ------------------------------------------------------------------------
\documentclass[dissertmst]{ppgco}
%Não altere o comando seguinte. O título de seu trabalho será especificado mais adiante.
\title{Template de Monografia do PPGCO}

% ---
% PACOTES
% ---

% ---
% Pacotes fundamentais 
% ---
\usepackage{cmap}				% Mapear caracteres especiais no PDF
\usepackage{lmodern}			% Usa a fonte Latin Modern			
\usepackage{makeidx}            % Cria o indice
\usepackage{hyperref}  			% Controla a formação do índice
\usepackage{lastpage}			% Usado pela Ficha catalográfica
\usepackage{indentfirst}		% Indenta o primeiro parágrafo de cada seção.
\usepackage{nomencl} 			% Lista de simbolos
\usepackage{graphicx}			% Inclusão de gráficos
% ---

% ---
% Pacotes adicionais, usados apenas no âmbito do Modelo eesc
% ---
\usepackage{lipsum}				       % para geração de dummy text
\usepackage[printonlyused]{acronym}
\usepackage[table]{xcolor}
% ---

% ---
% Pacotes pessoais
% ---
\usepackage{subfig}
%

% ---
% Informações de dados para CAPA e FOLHA DE ROSTO
% ---
%
% Título:
%	1. Título em português
%	2. Título em inglês
\titulo{Modeloaaaaaaa do Trabalho de Conclusão de Curso para o Bacharelado em Ciência da Computação}{Título em inglês}
%
% Autor:
%	1. Nome completo do autor
%	2. Formato de nome para bibliografia
\autor{Nome do Aluno}{Sobrenome, Primeiro Nome}
%
% Cidade
\local{Uberlândia}
% Ano de defesa
\data{2024}
% Área de concentração da pesquisa
\areaconcentracao{Ciência da Computação}
% Nome do orientador
\orientador{Dr. Claudiney Ramos Tinoco}
% Nome do coorientador
%\coorientador{Nome completo do coorientador}
% ---

% ---
% compila o indice
% ---
\makeindex
% ---

% ---
% Compila a lista de abreviaturas e siglas
% ---
\makenomenclature
% ---

% ---
% Inserir ficha catalográfica
%
% Caso o comando \inserirfichacatalografica seja definido, a %ficha catalográfica
% será inserida atrás da folha de rosto. Caso contrário a página será deixada em
% branco.
%
% CUIDADO: Esta opção deve ser preenchida antes do comando \maketitle
% ---
%entre em contato com a biblioteca para obter a sua ficha catalográfica em arquivo pdf. Essa %folha só será inserida no documento após a sua defesa.

% \inserirfichacatalografica{fichaCatalografica.pdf}
% ---

% ---
% Inserir folha de aprovação
%
% Caso o comando \inserirfolhaaprovacao seja definido, a a folha de aprovação
% será inserida. Além disso, conforme Resolução CoPGr 5890, as informações 
% de rodapé são inseridas apropriadamente na folha de rosto.
%
% CUIDADO: Esta opção deve ser preenchida antes do comando \maketitle
% ---
% baseie-se no modelo desse documento e gere a sua folha de %rosto em arquivo pdf.

% \inserirfolhaaprovacao{folhaAprovacao.pdf}
% ---

% ----
% Início do documento
% ----

\begin{document}

% ----------------------------------------------------------
% ELEMENTOS PRÉ-TEXTUAIS
% ----------------------------------------------------------
\pretextual

% ---
% Insere Capa, Folha de rosto, Ficha catalográfica (se inserida)
% e folha de aprovação (se inserida).
% ---
\maketitle


% ---
% Dedicatória
% ---
\imprimirdedicatoria{Este trabalho é dedicado às crianças adultas que,\\
   quando pequenas, sonharam em se tornar cientistas.}
% ---

% ---
% Agradecimentos
% ---
\imprimiragradecimentos{
Faça os agradecimentos àqueles que direta ou indiretamente contribuíram para que você tivesse obtido êxito. Inclua na sua lista agradecimentos aos órgãos de fomento, quando for o caso.
}
% ---

% ---
% Epígrafe
% ---
\imprimirepigrafe{
		``Sua vida pode ser dividida em dois períodos: antes de agora e a partir de agora.''\\
		(Prof. Obvious Stating)
}
% ---

% ---
% RESUMO e ABSTRACT
% ---

% Resumo em português - as palavras entre chaves são as palavras-chave do trbalho
\begin{resumo}{Latex. Abntex. Normas USP}
This is the portuguese abstract.
\end{resumo}

% Resumo em inglês
\begin{abstract}{Latex. Abntex.}
This is the english abstract.
\end{abstract}
% ---

% ---
% inserir lista de ilustrações
% ---
\listailustracoes
% ---

% ---
% inserir lista de tabelas
% ---
\listatabelas
% ---

% ---
% inserir lista de abreviaturas e siglas
% ---
\listasiglas{abrev/Abreviaturas}
% ---

% ---
% inserir o sumario
% ---
\sumario
% ---

% ----------------------------------------------------------
% ELEMENTOS TEXTUAIS
% ----------------------------------------------------------
\mainmatter

% ----------------------------------------------------------
% Introdução
% ----------------------------------------------------------
\include{cap_introducao/introducao}


% ----------------------------------------------------------
% Fundamentação
% ----------------------------------------------------------
\include{cap_fundamentacao/fundamentacao}

% ----------------------------------------------------------
% Proposta de pesquisa
% ----------------------------------------------------------
\include{cap_proposta/proposta}


% ----------------------------------------------------------
% Experimentos e avaliação dos resultados
% ----------------------------------------------------------

\include{cap_experimentos/experimentos}

% ---
% Finaliza a parte no bookmark do PDF, para que se inicie o bookmark na raiz
% ---
\bookmarksetup{startatroot}% 
% ---

% ---
% Conclusão
% ---
\include{cap_conclusao/conclusao}


 

% ----------------------------------------------------------
% ELEMENTOS PÓS-TEXTUAIS
% ----------------------------------------------------------
\postextual

% ----------------------------------------------------------
% Referências bibliográficas
% ----------------------------------------------------------
\bibliographystyle{bib/abntex2-alf}
\bibliography{bib/abntex2-references}

% ----------------------------------------------------------
% Glossário
% ----------------------------------------------------------
%
%\glossary

% ----------------------------------------------------------
% Apêndices
% ----------------------------------------------------------
% ---
% Inicia os apêndices
% ---
\begin{apendicesenv}
% Imprime uma página indicando o início dos apêndices
\partapendices
% ----------------------------------------------------------
% Incluir Apêndice
% ----------------------------------------------------------
% ----------------------------------------------------------
% Capitulo com exemplos de comandos inseridos de arquivo externo 
% ----------------------------------------------------------
%\include{ape_comandos/abntex2-modelo-include-comandos}

\end{apendicesenv}
% ---

% ----------------------------------------------------------
% Anexos
% ----------------------------------------------------------
% ---
% Inicia os anexos
% ---
\begin{anexosenv}
% Imprime uma página indicando o início dos anexos
\partanexos
% ---
% Incluir Anexo
% ---
%\chapter{Morbi ultrices rutrum lorem.}

%o comando lipsum[] comando serve apenas para incluir texto no documento para efeito de visualização do formato.
%\lipsum[1-25]
%\section{Test}
%\lipsum[1-20]


\end{anexosenv}


\end{document}